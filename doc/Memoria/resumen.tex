% TFG - José Ángel Martín Baos. Escuela Superior de Informática. 2018
% !TeX spellcheck = en_GB

\chapter{Abstract}

Road transport is the recognized major source of air pollution in urban areas, with detrimental effects on the local air quality, ecology, and even on human health. For this reason there is an increasing need to estimate precisely its contribution to air pollution on the cities. Dynamic Traffic Management (DTM) systems are used to reduce the negative externalities of the traffic congestion. Nonetheless, the use of these systems require reliable emissions models and traffic and environmental monitoring infrastructures. Moreover, cities are distributed environments where events occur in real time and on a massive scale. Hence, it is needed an intelligent system capable of monitoring environmental and traffic parameters using distributed sensors and which allows to predict the pollution levels and recommends several palliative actions.

This Bachelor of Science thesis aims to design and build a prototype of an inexpensive low-cost distributed system for monitoring the traffic flow and different environmental parameters (such as temperature, pressure, humidity and several pollutant gases). In addition, a web page has been developed where the data collected by the different sensors can be monitored in real-time. Moreover, the web page also contains an alert system to inform the transport authority if a concrete parameter get over a pre-defined threshold and an historical section where the data collected in a certain time interval is shown.


\chapter{\REDNOTE{Resumen}} % TODO

Resumen



