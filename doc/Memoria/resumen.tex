% TFG - José Ángel Martín Baos. Escuela Superior de Informática. 2018
% !TeX spellcheck = es_ES

\chapter{Abstract}

Road transport is the recognized major source of air pollution in urban areas, with detrimental effects on the local air quality, ecology, and even on human health. For this reason there is an increasing need to estimate precisely its contribution to air pollution on the cities. Dynamic Traffic Management (DTM) systems are used to reduce the negative externalities of the traffic congestion. Nonetheless, the use of these systems require reliable mathematical emissions models and traffic and environmental monitoring infrastructures. Moreover, cities are distributed environments where events occur in real time and on a massive scale. Hence, intelligent systems capable of monitoring environmental and traffic parameters using distributed sensors to predict the pollution levels and recommend several palliative actions are needed.

This Bachelor of Science thesis aims to design and build a prototype of a low-cost distributed Internet of Things (IoT) system for monitoring the traffic flow and different environmental parameters such as temperature, pressure, humidity and several pollutant gases. This information is uploaded to a cloud service where it is processed. In addition, a web page is developed where the data collected by the different sensors can be monitored in real-time. Moreover, the web page also contains an alert system to inform the transport authority if a concrete parameter get over a pre-defined threshold and a historical section where the data collected in a certain time interval is shown.


\chapter{Resumen}

El transporte por carretera es la principal fuente de contaminación atmosférica en las zonas urbanas, con efectos perjudiciales sobre la calidad del aire, la ecología e incluso sobre la salud humana. Por esta razón, hay una creciente necesidad de estimar con precisión su contribución a la contaminación del aire en las ciudades. Los sistemas de Gestión Dinámica del Tráfico (DTM por sus siglas en inglés) se utilizan para la gestión y control del tráfico, y así poder reducir las externalidades negativas de la congestión. No obstante, el uso de estos sistemas requiere de modelos matemáticos confiables de emisiones y de infraestructuras de control de tráfico y de medio ambiente. Además, las ciudades son entornos distribuidos donde los eventos ocurren en tiempo real y de manera masiva. Por ello, se necesitan sistemas inteligentes capaces de monitorizar los parámetros ambientales y de tráfico utilizando sensores distribuidos para predecir los niveles de contaminación y recomendar diferentes acciones paliativas.

Este Trabajo Fin de Grado (TFG) tiene como objetivo el diseño y construcción de un prototipo de sistema distribuido del Internet de las Cosas (IoT por sus siglas en inglés) de bajo coste que permita monitorizar el flujo de tráfico y diferentes parámetros ambientales tales como la temperatura, presión, humedad y varios gases contaminantes. Esta información se envía a un servicio en la nube donde será procesada. Además, se ha desarrollado una página web donde se muestran en tiempo real los datos recopilados por los diferentes sensores. Esta página web también contiene un sistema de alerta para informar a la autoridad de transporte si un parámetro concreto supera un umbral predefinido, así como una sección donde los datos recopilados en un intervalo de tiempo son mostrados de forma gráfica.



