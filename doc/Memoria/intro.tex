% TFG - José Ángel Martín Baos. Escuela Superior de Informática. 2018

%%% CHAPTER: Introduction %%%
\chapter{Introduction}
\drop{R}{oad} transportation has become the main source of air pollution cities and urban areas, which has a big impact in local air quality and human health. For this reason, there is an increasing need to estimate precisely the contribution of road transport to air pollution in the cities, so that pollution-reduction measures can be designed and implemented appropriately \cite{SNB10}. These pollution-reduction measures are becoming increasingly relevant due to the continued growth in vehicle use and the deterioration in driving conditions (traffic congestion). Many authorities find difficult to meet their environmental targets (e.g. air quality standards or national emission ceilings). Therefore, reliable emission models are needed in order to predict accurately the impact of road transport on air pollution.

Nowadays, intelligent cities are essential to prevent situations of high level contamination and to take measures when these situations occur. Cities should predict the pollution peaks and take palliative measures, such as restricting the traffic to a certain number of vehicles or depending on of their the license plates, closing traffic in certain streets, lowering the speed limits, etc. Moreover, traffic flows should be monitored as they affect the pollution levels in that city. For this reason, cities should rely on an \ac{IoT} infrastructure connected to a cloud platform that supports these kind of systems as well as sensor-based big data applications \cite{Bib18}.

Typical pollution surveillance and control systems are usually composed of big and expensive devices that are limited only to some points in the city, hence, they provide information for vast areas and sometimes these systems are not scalable. However, cities are distributed environments where events occur in real time and on a massive scale. Therefore, an inexpensive distributed \ac{IoT} architecture is recommendable to control the pollution levels by neighbourhoods or streets. These pollution control systems can be combined with a traffic surveillance infrastructure in order to have a complete system that could be used as a \ac{DSS} to help authorities to take decisions about environmental impacts caused by pollution before they occur.

The main objective in this \ac{BSc.} thesis is to design and build a prototype of an integrated low-cost road traffic and air pollution monitoring platform. It focuses on the design and implementation of the software and hardware architecture that should allow the future implementation of an intelligent system for the prediction of the pollution levels, the recommendation of palliative actions, and the monitoring of the taken actions. An inexpensive embedded system is used as architecture in order to produce a scalable system.

Figure \ref{fig:1-system_architecture} represents the architecture proposed in this \ac{BSc.} thesis. We have focused on the monitoring infrastructure, which is composed of devices that on the one hand monitors environmental parameters such as temperature, pressure, humidity and different pollutant gases, whilst on the other hand monitors traffic parameters such as the vehicle flow, speed or flow density. The environmental parameters are monitored by means of several sensors installed on the device, whereas the traffic parameters are obtained by processing a video captured by a camera integrated in the own device. \\ \\

\begin{figure}[!h]
	\begin{center}
		\includegraphics[width=0.90\textwidth]{1-system_architecture.pdf}	
		\caption{Prototype of the pollution surveillance and traffic control system}
		\label{fig:1-system_architecture}
	\end{center}
\end{figure}


\newpage
\section{Document structure}

This section describes how the rest of the document is organized. To this end, each one of the subsequent chapters is briefly presented.

\begin{definitionlist}
	\item[Chapter \ref{chap:objectives}: \nameref{chap:objectives}] In this chapter the different general and specific objectives that are addressed in this work are defined.
	
	\item[Chapter \ref{chap:background}: \nameref{chap:background}] For the development of this \ac{BSc.} thesis a bibliographic review has been carried out. Firstly, an introduction about traffic flows and traffic emissions monitoring systems is discussed. In addition, some European projects which aims to measure and control the pollution generated are described. Secondly, an introduction is made about embedded systems and how they can be implemented in traffic and environmental surveillance. Raspberry Pi embedded systems are going to be used in this thesis, therefore their features and uses are described here. To finish with, it will go on to explain H.264/AVC video format, which is used to measure the traffic flow using a camera.
	
	\item[Chapter \ref{chap:methodology}: \nameref{chap:methodology}] In this chapter the working methodology used to develop this \ac{BSc.} thesis is explained. For this purpose, Scrum has been used as project management methodology and Kanban to control the progress of the project. In addition, the iterative and incremental software development methodology is used. To end with, the different physical and software resources required to perform this work are defined.
	
	\item[Chapter \ref{chap:results}: \nameref{chap:results}] In this chapter the results and artefacts derived from the working plan are presented. This chapter is divided into different Sprints defined by using the Scrum methodology.
	
	\item[Chapter \ref{chap:conclusions}: \nameref{chap:conclusions}] In this chapter the main milestones achieved during the execution of this project are summarised. In addition, a set of improvements and proposals for future work are commented. 
	%\REDNOTE{It will go on to make a brief reflection about the knowledge acquired during the realization of this work.}
	
	\item[Appendix \ref{chap:installation_guide}: \nameref{chap:installation_guide}] In this appendix the different steps needed to install and configure Raspberry Pi are described. It includes the installation of the Raspbian Operating System in the Raspberry Pi, as well as the installation of the different sensors and libraries.
	
	\item[Appendix \ref{chap:config_file}: \nameref{chap:config_file}] In this appendix the configuration file used by the Raspberry Pi device is shown.
\end{definitionlist}


%%% CHAPTER: Introducción %%%
\chapter{Introducción}
\drop{E}{l} transporte por carretera se ha convertido en la principal fuente de contaminación del aire en ciudades y áreas urbanas, lo que tiene un gran impacto en la calidad del aire y la salud humana. Por esta razón, existe una creciente necesidad de estimar con precisión la contribución del transporte por carretera a la contaminación del aire en las ciudades, de tal forma que se puedan diseñar e implementar de forma adecuada medidas para reducir esta contaminación \cite{SNB10}. Estas medidas son cada vez más necesarias debido al continuo crecimiento del uso de los vehículos y al incremento de la congestión del tráfico. Muchas autoridades encuentran difícil cumplir sus objetivos medioambientales estipulados mediante normas de calidad del aire o límites nacionales máximos de emisión. Por lo tanto, se necesitan modelos de emisión confiables para predecir con precisión el impacto del transporte por carretera en la contaminación del aire.

Hoy en día, las ciudades inteligentes son esenciales para prevenir situaciones de altos niveles de contaminación y para tomar medidas cuando estas situaciones ocurran. Las ciudades deben predecir los picos de contaminación y tomar medidas paliativas, como restringir el tráfico a un cierto número de vehículos, o dependiendo de sus matrículas, cerrar el tráfico en ciertas calles, reducir los límites de velocidad, etc. Además, los flujos de tráfico deben ser monitorizados, ya que afectan a los niveles de contaminación de la ciudad. Por este motivo, las ciudades deberían disponer de una infraestructura \ac{IoT} (Internet de las Cosas) conectada a una plataforma en la nube que admita este tipo de sistemas, así como aplicaciones de \emword{big data} basadas en sensores \cite{Bib18}.

Los sistemas de vigilancia y control de la contaminación suelen estar compuestos de dispositivos grandes y caros colocados sólo en algunos puntos de la ciudad. Por lo tanto, estos dispositivos proporcionan información para áreas extensas y normalmente no son escalables. Sin embargo, las ciudades son entornos distribuidos donde los eventos ocurren en tiempo real y en una escala masiva. Por esta razón, se recomienda una arquitectura \ac{IoT} distribuida de bajo coste para controlar los niveles de contaminación por barrios o calles. Además, este sistema puede combinarse con una infraestructura de vigilancia del tráfico para tener un sistema completo que podría usarse como un sistema de soporte a la toma de decisiones (\ac{DSS} por sus siglas en inglés), para ayudar a las autoridades a tomar medidas sobre los impactos ambientales que puede causar la contaminación antes de que ocurran.

El principal objetivo en este \ac{TFG} es diseñar y construir un prototipo de plataforma integrada de bajo coste para monitorizar el tráfico y la contaminación del aire. Este trabajo se enfoca en el diseño e implementación de una arquitectura software y hardware que permita la el despliegue futuro de un sistema inteligente para la predicción de los niveles de contaminación, la recomendación de acciones paliativas y el monitoreo de las acciones tomadas. Se usa un sistema empotrado de bajo coste como arquitectura para producir un sistema escalable.

La Figura \ref{fig:2-arquitectura_sistema} representa la arquitectura propuesta en este \ac{TFG}. Nos hemos centrado en la infraestructura de monitorización, que se compone de dispositivos que, por un lado, monitorizan parámetros ambientales como la temperatura, presión, humedad y diferentes gases contaminantes; mientras que, por otro lado, controlan los parámetros de tráfico, tales como el flujo de vehículos, la velocidad de estos o la densidad del flujo. Los parámetros ambientales se controlan mediante varios sensores instalados en el dispositivo, mientras que los parámetros de tráfico se obtienen procesando la señal capturada por una cámara integrada en el propio dispositivo.

\begin{figure}[!h]
	\begin{center}
		\includegraphics[width=0.90\textwidth]{2-arquitectura_sistema.pdf}	
		\caption{Prototipo del sistema de vigilancia de tráfico y contaminación}
		\label{fig:2-arquitectura_sistema}
	\end{center}
\end{figure}


\section{Estructura del documento} 
En esta sección se describe como está organizado el resto del documento. Para ello se presenta brevemente cada uno de los capítulos posteriores.

\begin{definitionlist}
	\item[Capítulo \ref{chap:objectives}: Objetivos] En este capitulo se definen los diferentes objetivos generales y específicos que serán tratados en este \ac{TFG}.
	
	\item[Capítulo \ref{chap:background}: Antecedentes] Para el desarrollo de este \ac{TFG} se ha realizado una búsqueda bibliográfica. En este capítulo, primero se introducen conceptos sobre los flujos de tráfico y los sistemas de monitorización de emisiones debidas al tráfico. Además, se detallan algunos proyectos Europeos que tratan de medir y controlar la contaminación generada. En segundo lugar, se realiza una introducción sobre los sistemas empotrados y cómo pueden ser utilizados para monitorizar el tráfico y los parámetros medioambientales. A continuación, se describen los sistemas empotrados Raspberry Pi, que serán usados en este \ac{TFG}. Por último, se explica el formato de video H.264/AVC, que es utilizado para medir el flujo de tráfico usando la cámara integrada en el dispositivo.
	
	\item[Capítulo \ref{chap:methodology}: Metodología] En este capítulo se expone la metodología de trabajo usada para el desarrollo del \ac{TFG}. Para ello, Scrum ha sido utilizado como metodología de gestión de proyectos y Kanban para controlar el progreso de este. Además, se ha utilizado una metodología de desarrollo software iterativa e incremental. Para finalizar, los diferentes medios hardware y software utilizados para el desarrollo del \ac{TFG} son explicados. 
	
	\item[Capítulo \ref{chap:results}: Resultados] En este capítulo se explican los diferentes resultados y artefactos obtenidos del desarrollo del plan de trabajo propuesto en el capítulo anterior. Este capitulo está dividido en diferentes Sprints definidos usando la metodología Scrum.
	
	\item[Capítulo \ref{chap:conclusiones}: Conclusiones] En este capítulo se explican los principales hitos conseguidos durante la ejecución de este proyecto. Además, se comentan un conjunto de mejoras y propuestas de trabajo futuro.
	
	\item[Anexo \ref{chap:installation_guide}: Guía de instalación] En este Anexo se describen los diferentes pasos para instalar y configurar el dispositivo desarrollado. Esto incluye la instalación del sistema operativo Raspbian y de los diferentes sensores y bibliotecas.
	
	\item[Anexo \ref{chap:config_file}: Archivo de configuración] En este anexo se muestra el archivo de configuración usado por el dispositivo Raspberry Pi.
	
\end{definitionlist}
