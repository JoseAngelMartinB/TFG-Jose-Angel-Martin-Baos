% TFG - José Ángel Martín Baos. Escuela Superior de Informática. 2018
%%%% CHAPTER: Introducción %%%

%%% CHAPTER: Introduction %%%
\chapter{Introduction} % TODO: Revise
\drop{R}{oad} transportation is often the main source of air pollution in urban areas with detrimental effects on local air quality, ecology, and human health. Therefore, there is an increasing need to estimate precisely the contribution of road transport to air pollution in the cities, so that pollution-reduction measures can be designed and implemented appropriately \cite{SNB10}. These pollution-reduction measures are becoming increasingly relevant due to the continued growth in vehicle use and the deterioration in driving conditions (traffic congestion). Many authorities find it difficult to meet their environmental targets (e.g. air quality standards or national emission ceilings) and, therefore, reliable emission models are needed, in order to predict accurately the impact of road transport on air pollution.

Therefore, intelligent cities are essential to prevent situations of high level contamination and take measures when these situations occur. These cities must predict the pollution peaks and take palliative measures, such as restricting the traffic to a number of vehicles, by the license plates, closing traffic in some streets, lowering the speed limits, etc. Moreover, traffic flows must be monitored as they affect the pollution levels in that city. For this reason, these cities must rely on an \ac{IoT} infrastructure connected to a cloud platform that supports those systems as well as sensor-based big data applications. \cite{Bib18}.

Typical pollution surveillance and control systems are composed of big and expensive devices that are only located in some points in a city, hence, they provide information for vast areas and sometimes those systems are not scalable. However, cities are distributed environments where the events occur in real time and on a massive scale. Therefore, an inexpensive distributed IoT architecture is needed to control the pollution levels by zones or streets. These pollution control systems can be combined with a traffic surveillance infrastructure in order to have a complete system that could be used as a \ac{DSS} to help authorities to make decisions about environmental problems caused by pollution before they occur.

The objective in this \ac{BSc.} thesis is to design and build a prototype of an integrated low-cost road traffic and air pollution monitoring platform. It only focuses on the design and implementation of the software and hardware architecture that must allow the future implementation of an intelligent system for the prediction of the pollution levels, the recommendation of palliative actions and monitoring those actions. An inexpensive embedded system is used for designing the architecture in order to obtain a scalable system not only in size but also in cost.

Figure \ref{fig:2-system_architecture} represents the architecture purposed in this \ac{BSc.} thesis. We have focused on the monitoring infrastructure, which is composed of devices that monitor environmental parameters such as temperature, pressure, humidity and different pollutant gases; and traffic parameters such as the vehicle flow, speed or flow density. The environmental parameters are monitored by means of several sensors installed on the device, whereas the traffic parameters are obtained by processing a video captured by a camera integrated in the device.


\begin{figure}[!h]
	\begin{center}
		\includegraphics[width=1\textwidth]{2-system_architecture.pdf}	
		\caption{A prototype of pollution surveillance and traffic control system}
		\label{fig:2-system_architecture}
	\end{center}
\end{figure}

%TODO: La introducción debe ser una presentación de los objetivos, en la cual se introduzcan todos los conceptos que aparezcan en la sección de objetivos.

% Contar que se va a usar el video en tiempo real para detectar los coches

% Hablar sobre el cloud

% Hablar sobre los terminos del O.4

\newpage
\section{Document structure}

This section describes how the rest of the document is organized. To this end, each of the subsequent chapters is briefly presented.

\begin{definitionlist}
	\item[Chapter \ref{chap:objectives}: \nameref{chap:objectives}] In this chapter the different general and specific objectives that are addressed in this work are defined.
	
	\item[Chapter \ref{chap:background}: \nameref{chap:background}] For the development of this \ac{BSc.} thesis a bibliographic review has been carried out. Firstly, an introduction about traffic flows and traffic emissions monitoring systems is discussed. In addition, some European projects which aims to measure and control de pollution generated are described. Secondly, an introduction is made about embedded systems and how they can be implemented in traffic and environmental surveillance. Raspberry Pi embedded systems are going to be used in this thesis, therefore their features and uses are described here. To finish with, it will go on to explain H.264/AVC video format, which is going to be used to measure the traffic flow using a camera.
	
	\item[Chapter \ref{chap:methodology}: \nameref{chap:methodology}] In this chapter the working methodology used to develop this \ac{BSc.} thesis is explained. To this purpose, Scrum has been used as project management methodology and Kanban to control the progress of the project. In addition, the iterative and incremental software development methodology is used. To end with, the different physical and software resources required to perform this work are defined.
	
	\item[Chapter \ref{chap:results}: \nameref{chap:results}] In this chapter the results and artefacts derived from the working plan are presented. This chapter is divided into different Sprints defined by using the Scrum methodology.
	
	\item[Chapter \ref{chap:conclusions}: \nameref{chap:conclusions}] In this chapter the main milestones achieved during the execution of this project are summarised. In addition, a set of improvements and proposals for future work are commented. 
	%\REDNOTE{It will go on to make a brief reflection about the knowledge acquired during the realization of this work.}
	
	\item[Appendix \ref{chap:installation_guide}: \nameref{chap:installation_guide}] In this appendix the different steps needed to install and configure Raspberry Pi are described.  Appendix \nameref{chap:installation_guide}] includes the installation of the Raspbian Operating System in the Raspberry Pi, as well as the installation of the different sensors and libraries.
	
	\item[Appendix \ref{chap:config_file}: \nameref{chap:config_file}] In this appendix the configuration file used by the different Raspberry Pi devices is shown.
\end{definitionlist}



\chapter{\REDNOTE{Introducción}} % TODO
%%%%%
% \drop{I}{n}


\REDNOTE{TRADUCIR SECCIÓN ANTERIOR}

\section{Estructura del documento}
En esta sección se describe como está organizado el resto del documento. Para ello se presenta brevemente cada uno de los capítulos posteriores.

\begin{definitionlist}
	\item[Capítulo \ref{chap:objectives}: \nameref{chap:objectives}] \REDNOTE{Explicar...}
	
	\item[Capítulo \ref{chap:background}: \nameref{chap:background}] \REDNOTE{Explicar...}
	
	\item[Capítulo \ref{chap:methodology}: \nameref{chap:methodology}] \REDNOTE{Explicar...}
	
	\item[Capítulo \ref{chap:results}: \nameref{chap:results}] \REDNOTE{Explicar...}
	
	\item[Capítulo \ref{chap:conclusiones}: \nameref{chap:conclusiones}] \REDNOTE{Explicar...}
	
	\item[Anexo \ref{chap:installation_guide}: \nameref{chap:installation_guide}] \REDNOTE{Explicar...}
	
	\item[Anexo \ref{chap:config_file}: \nameref{chap:config_file}] \REDNOTE{Explicar...}
	
\end{definitionlist}
