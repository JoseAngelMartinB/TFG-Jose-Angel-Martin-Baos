% TFG - José Ángel Martín Baos. Escuela Superior de Informática. 2018
%%%% CHAPTER: Introducción %%%

%%% CHAPTER: Introduction %%%
\chapter{Introduction} % TODO: Revise
\drop{R}{oad} transportation is often the main source of air pollution in urban areas with detrimental effects on local air quality, ecology, and human health. Therefore, there is an increasing need to estimate precisely the contribution of road transport to air pollution in the cities, so that pollution-reduction measures can be designed and implemented appropriately \cite{SNB10}. These pollution-reduction measures are becoming increasingly relevant due to the continued growth in vehicle use and the deterioration in driving conditions (congestion). Many authorities find it difficult to meet their environmental targets (e.g. air quality standards or national emission ceilings) and, therefore, reliable emission models are needed, in order to predict accurately the impact of road transport on air pollution.

Therefore, intelligent cities are essential to prevent situations of high level contamination and take measures when these situations occur. These cities must predict the pollution peaks and take palliative measures, such as restricting the traffic to a number of vehicles, by the license plates, closing traffic in some streets, lowering the speed limits, etc. Moreover, traffic flows must be monitored as they affect the pollution levels in that city. For this reason, these cities must rely on an \ac{IoT} infrastructure connected to a cloud platform that supports those systems as well as sensor-based big data applications. \cite{Bib18}.

Typical pollution surveillance and control systems are composed of big and expensive devices that are only located in some points in a city, hence, they provide information for vast areas and sometimes those systems are not scalable. However, cities are distributed environments where the events occur in real time and on a massive scale. Therefore, an inexpensive distributed IoT architecture is needed to control pollution levels by zones or streets. These can be combined with a traffic surveillance infrastructure in order to have a complete system that could be used as a Decision Support System (DSS) to help authorities to take decisions about environmental problems caused by pollution before they occur.

The objective in this \ac{BSc.} thesis is to design and build a prototype of an integrated low-cost road traffic and air pollution monitoring platform. It will only focus on the design and implementation of the software and hardware architecture that must allow the future implementation of an intelligent system for the prediction of the pollution levels, the recommendation of palliative actions and monitoring those actions. An inexpensive embedded system will be used for designing the architecture in order to obtain a scalable system not only in size but also in cost.


\begin{figure}[!h]
	\begin{center}
		\includegraphics[width=1\textwidth]{2-system_architecture.pdf}	
		\caption{System architecture}
		\label{fig:2-system_architecture}
	\end{center}
\end{figure}




\section{Document structure}

This section describes how the rest of the document is organized. To this end, each of the subsequent chapters is briefly presented.

\begin{definitionlist}
	\item[Chapter \ref{chap:objectives}: \nameref{chap:objectives}] In this chapter the different general and specific objectives that will be addressed in this work are defined.
	
	\item[Chapter \ref{chap:background}: \nameref{chap:background}] For the development of this \ac{BSc.} thesis a bibliographic research has been carried out. Firstly, an introduction about traffic flows and traffic emissions monitoring systems is discussed. In addition, some European projects which aims to measure and control de pollution generated are described. Then, an introduction is made about embedded systems and how they can be implemented in traffic or environmental surveillance. Raspberry Pi embedded systems are going to be used in this thesis, therefore their features and uses are described here. To finish with, it will go on to explain H.264/AVC video format, which is going to be used to measure the traffic flow using a camera.
	
	\item[Chapter \ref{chap:methodology}: \nameref{chap:methodology}] In this chapter the working methodology used to develop this \ac{BSc.} thesis is explained. To this effect, Scrum has been used as project management methodology and Kanban to control the progress of the project. In addition, the iterative and incremental software development methodology is used. To end with, the different physical and software resources required to perform this work are defined.
	
	\item[Chapter \ref{chap:results}: \nameref{chap:results}] In this chapter the work planning is presented. Subsequently, the different results and artefacts obtained from applying the methodology presented in the previous chapter, as well as the inconveniences produced during the project realization are described. 
	
	\item[Chapter \ref{chap:conclusions}: \nameref{chap:conclusions}] In this chapter the main milestones achieved during the execution of this project are described. In addition, a set of improvements and proposals for future work are commented. 
	%\REDNOTE{It will go on to make a brief reflection about the knowledge acquired during the realization of this work.}
	
	\item[Appendix \ref{chap:installation_guide}: \nameref{chap:installation_guide}] In this appendix the different steps needed to install and configure Raspberry Pi are described. This Appendix include the installation of the Raspbian Operating System in the Raspberry Pi, as well as the installation of the different sensors and libraries.
	
	\item[Appendix \ref{chap:config_file}: \nameref{chap:config_file}] In this appendix the configuration file used by the different Raspberry Pi devices is described.
\end{definitionlist}



\chapter{Introducción} % TODO
%%%%%
% \drop{I}{n}


\section{Estructura del documento}
En esta sección se describe como está organizado el resto del documento. Para ello se presenta brevemente cada uno de los capítulos posteriores.

\begin{definitionlist}
	\item[Capítulo \ref{chap:objectives}: \nameref{chap:objectives}] En este capítulo los diferentes objetivos generales y específicos que se pretenden abordar en este trabajo son definidos.
	
	\item[Capítulo \ref{chap:background}: \nameref{chap:background}] \REDNOTE{Explicar...}
	
	\item[Capítulo \ref{chap:methodology}: \nameref{chap:methodology}] En este capítulo se describe la metodología de trabajo seguida para desarrollar este \ac{TFG}. Para ello, se ha utilizado Scrum como metodología de gestión de proyecto y Kanban para controlar el avance del proyecto. Además, se definen los distintos medios físicos y medios software necesarios para realizar este trabajo.
	
	\item[Capítulo \ref{chap:results}: \nameref{chap:results}] En este capítulo se muestra primero la planificación de trabajo que se ha realizado. Posteriormente, se mostrarán los distintos resultados y artefactos obtenidos a partir de aplicar la metodología expuesta en el capítulo anterior, así como los inconvenientes producidos durante la realización del proyecto.
	
	\item[Capítulo \ref{chap:conclusiones}: \nameref{chap:conclusiones}] En el capítulo se hace una conclusión donde se indican los principales hitos conseguidos durante la realización de este proyecto. Ademas, se plantean una conjunto de mejoras y propuestas de trabajo futuro en el ámbito de este \ac{TFG}. \REDNOTE{Por último, se realiza una breve reflexión sobre los conocimientos adquiridos durante la realización de este trabajo.}
	
	\item[Anexo \ref{chap:installation_guide}: \nameref{chap:installation_guide}] En este anexo se describen los pasos necesarios para instalar y configurar la Raspberry Pi de manera correcta. Esto incluye, la instalación del Sistema Operativo Raspbian sobre la Raspberry Pi, así como la instalación de los distintos periféricos y librerías.
	
	\item[Anexo \ref{chap:config_file}: \nameref{chap:config_file}] \REDNOTE{Explicar...}
	
\end{definitionlist}
