% TFG - José Ángel Martín Baos. Escuela Superior de Informática. 2017
%%%% CHAPTER: Introducción %%%
\chapter{Introducción} % TODO
	
% \drop{I}{n}
Introducción


\section{Motivación}
% Necesidad del trabajo realizado en la sociedad actual.


\section{Estructura del documento}
En esta sección se describe como está organizado el resto del documento. Para ello se presenta brevemente cada uno de los capítulos posteriores.

\begin{definitionlist}
	\item[Capítulo \ref{chap:objectives}: \nameref{chap:objectives}] En este capítulo los diferentes objetivos generales y específicos que se pretenden abordar en este trabajo son definidos.
	
	\item[Capítulo \ref{chap:state_of_the_art}: \nameref{chap:state_of_the_art}] \REDNOTE{Explicar...}
	
	\item[Capítulo \ref{chap:methodology}: \nameref{chap:methodology}] En este capítulo se describe la metodología de trabajo seguida para desarrollar este \ac{TFG}. Para ello, se ha utilizado Scrum como metodología de gestión de proyecto y Kanban para controlar el avance del proyecto. Además, se definen los distintos medios físicos y medios software necesarios para realizar este trabajo.
	
	\item[Capítulo \ref{chap:results}: \nameref{chap:results}] En este capítulo se muestra primero la planificación de trabajo que se ha realizado. Posteriormente, se mostrarán los distintos resultados y artefactos obtenidos a partir de aplicar la metodología expuesta en el capítulo anterior, así como los inconvenientes producidos durante la realización del proyecto.
	
	\item[Capítulo \ref{chap:conclusiones}: \nameref{chap:conclusiones}] En el capítulo se hace una conclusión donde se indican los principales hitos conseguidos durante la realización de este proyecto. Ademas, se plantean una conjunto de mejoras y propuestas de trabajo futuro en el ámbito de este \ac{TFG}. \REDNOTE{Por último, se realiza una breve reflexión sobre los conocimientos adquiridos durante la realización de este trabajo.}
	
	\item[Anexo \ref{chap:installation_guide}: \nameref{chap:installation_guide}] En este anexo se describen los pasos necesarios para instalar y configurar la Raspberry Pi de manera correcta. Esto incluye, la instalación del Sistema Operativo Raspbian sobre la Raspberry Pi, así como la instalación de los distintos periféricos y librerías.
	
	\item[Anexo \ref{chap:user_manual}: \nameref{chap:user_manual}] \REDNOTE{Explicar...}
	
\end{definitionlist}




%%% CHAPTER: Introduction %%%
\chapter{Introduction} % TODO

Introduction


\section{Motivation}



\section{Document structure}

This section describes how the rest of the document is organized. To this end, each of the subsequent chapters is briefly presented.

\begin{definitionlist}
	\item[Chapter \ref{chap:objectives}: \nameref{chap:objectives}] In this chapter the different general and specific objectives that will be addressed in this work are defined.
	
	\item[Chapter \ref{chap:state_of_the_art}: \nameref{chap:state_of_the_art}] \REDNOTE{Explain...}
	
	\item[Chapter \ref{chap:methodology}: \nameref{chap:methodology}] In this chapter the working methodology used to develop this \ac{TFG} is explained. To this effect, Scrum has been used as project management methodology and Kanban to control the progress of the project. In addition, the different physical and software resources required to perform this work are defined.
	
	\item[Chapter \ref{chap:results}: \nameref{chap:results}] In this chapter first shows the work planning. Subsequently, the different results and artefacts obtained form applying the methodology presented in the previous chapter, as well as the inconveniences produced during the project realization will be shown. 
	
	\item[Chapter \ref{chap:conclusions}: \nameref{chap:conclusions}] In the chapter the main milestones achieved during the realization of this project are listed. In addition, a set of improvements and proposals for future work are proposed in the scope of this \ac{TFG}. \REDNOTE{The chapter ends with a brief reflection about the knowledge acquired during the realization of this work.}
	
	\item[Appendix \ref{chap:installation_guide}: \nameref{chap:installation_guide}] In this appendix the different steps needed to install and configure Raspberry Pi are stated. This include, the installation of the Raspbian Operative System in the Raspberry Pi, as well as the installation of the different peripherals and libraries.
	
	\item[Appendix \ref{chap:user_manual}: \nameref{chap:user_manual}] \REDNOTE{Explain...}
\end{definitionlist}


% Local Variables:
%  coding: utf-8
%  mode: latex
%  mode: flyspell
%  ispell-local-dictionary: "castellano8"
% End:
