% TFG - José Ángel Martín Baos. Escuela Superior de Informática. 2017
%%%% CHAPTER: State of the Art %%%
\chapter{State of the Art} % TODO
\label{chap:state_of_the_art}

% \drop{E}{ste} --> letra capital
State of the Art. También llamado
Antecedentes - Backgrounds
% TODO --> Antecedentes en ingles es Backgrounds 


%Frases interesantes:
% It will go on to -inf ----



Referencias:
https://www.raspberrypi.org/blog/vectors-from-coarse-motion-estimation/



% Apartados a tratar:
% Introducción a los fundamentos teóricos sobre los que se basa el TFG.
% Raspberry Pi 3 	-> Evolución
%					-> Comparativa con sus competidores
% Codificación de vídeo -> H.264/AVC
% Soluciones Similares a las desarrolladas en este TFG.




%TODO: Explicar en algún apartado del tfg porque se usa Raspbian y de donde surge.


% Local Variables:
%  coding: utf-8
%  mode: latex
%  mode: flyspell
%  ispell-local-dictionary: "castellano8"
% End:
