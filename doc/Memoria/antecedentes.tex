% TFG - José Ángel Martín Baos. Escuela Superior de Informática. 2018
%%%% CHAPTER: State of the Art %%%
% !TeX spellcheck = en_GB

\chapter{State of the Art}
\label{chap:state_of_the_art}

\drop{T}{his} chapter aims to explain some important concepts that will be used as the base for the development of this \ac{BSc.} thesis. Firstly, an introduction about traffic flows and traffic emissions monitoring is discussed. Then, embedded systems are explained, along with some of these systems which are used for solving this problem. It will go on to introduce Raspberry Pi systems and how they can be used in order to achieve the objectives of this \ac{BSc.} Thesis.
%as well as, some techniques used in this \ac{BSc.} Thesis such as H.264/AVC video codec.

\section{Traffic pollution}
% Introducción general a la problemática, a los sitemas actuales (proyecto europeo). Mencionar los contaminantes a medir. 
% Referencias: (consultar)
%	- Estimación espacio/temporal de la contaminación urbana asociada al tráfico: aplicación a la ciudad de México
%	- Traffic data for local emissions monitoring at a signalized intersection
% 	- Validation of road vehicle and traffic emission models
%	- Modelling instantaneous traffic emission and the influence of traffic speed limits
%	- European project --> https://ec.europa.eu/clima/policies/transport_en
%	- Air quality in Europe — 2016 report

There is an increasing need to estimate precisely the contribution of road transport to air pollution in the cities as a result of the fact that it is often the main source of air pollution in urban areas. For this reason, many authorities have developed some emissions models and systems in order to predict the road transport contribution to air pollution. Some systems, such as \ac{DTM} systems can focus on lower local emissions, using tools such as variable speed limits, ramp metering, adaptive signal timing \cite{MK10}, vehicle-class routing or prioritization \cite{ZDHB09}. These measures can generate some secondary effects such as longer travel delays, a decrease in the transit performance, or higher greenhouse gas emissions, therefore they should only be activated when they are warranted by air quality conditions \cite{EMA09} and, again, reliable emissions models are needed for this purpose. 

\subsection{Traffic pollution in Europe}
In Europe, transport represents a quarter of it's greenhouse gas emissions. Whereas the pollution generated by other sectors have been decreasing during the last years, the transport sector emissions only started to decrease in 2007 and they still remain higher than in 1990, as Figure \ref{fig:4-Emissions-Europe-2016} shows. Consequently, in July 2016, the European Commission adopted a low-emission mobility strategy, which aims to ensure that Europe stay competitive and it is able to respond the increasingly mobility needs of people and goods \cite{EuStrat}. 

% TODO - 
%
%
% ---------------------------------------
% Talk about AirQualityEEA16 and explain the main Gases
%
% MQ-2 -> CH4 (methane)
% MQ-7 -> CO


\REDNOTE{Talk about AirQualityEEA16 and explain the main Gases} 

\REDNOTE{QUÉ GASES Y PARAMETROS AMBIENTALES SE HAN ELEGIDO Y POR QUË!!!!!!!!!!!1}

\cite{AirQualityEEA16}

\begin{figure}[!h]
	\begin{center}
		\includegraphics[width=1\textwidth]{4-Emissions-Europe-2016.png}	
		\caption{Evolution of greenhouse gas emissions by sector (1990=100)}{Source: \ac{EEA}}
		\label{fig:4-Emissions-Europe-2016}
	\end{center}
\end{figure}


% http://www.airqualitynow.eu/es/about_indices_definition.php
Some projects like \ac{CITEAIR} has been created by the European Union \cite{citeair} in order to develop efficient means to collect, present and compare air quality data across multiple European cites. One of the results of this project was the craation of the \emph{Air quality now} webpage \cite{airqualitynow}. This page provides a platform to compare real time air quality measurements in different cities of Europe. In this page, air quality is measured in three different time scales: hourly index during the last day, daily index of the previous day and an annual index. Each of this three time scales has 5 indices using a pollution scale that goes from 0 (very low) to 100 (very high). 6 pollutants are measured, these are (PM10, NO2, O3, CO, PM2.5 and SO2). Then, the index is calculated depending on the amount of particles per hour (measured in $\mu g/m^3$), except for CO, in which a 8 hours moving average is used. Figure \ref{fig:4-Pollution-Index-Airqualitynow} shows the relationship between the pollutant measurement and its pollution index assigned.

\begin{figure}[!h]
	\begin{center}
		\includegraphics[width=1\textwidth]{4-Pollution-Index-Airqualitynow.png}	
		\caption{Relationship between the pollutant measurement and its pollution index assigned}{Source: Air Quality Now webpage \cite{airqualitynow}}
		\label{fig:4-Pollution-Index-Airqualitynow}
	\end{center}
\end{figure}

% Mencionar ciudades como casos de la problemática.
With the goal of evidencing the problem of traffic pollution in big cities, some examples has been selected. Air quality information about Madrid, Berlin and Paris taken on Tuesday 6th of February 2018 are shown in Figure \ref{fig:4-AirQuality-Details}.
As it can be observed, the pollution indices that day were very high, especially in some cities such as Berlin, where its current roadside pollution level has high (index value between 75 and 100). If the yesterday roadside indices values are compared, it can be shown that in all the selected cities its pollution indices are higher than 50, which means that some measures must be taken, especially in Berlin. This evidence reinforces the fact that some pollution prevention and control mechanisms are necessary.

\begin{figure}[htb]
	\centering
	\subfigure[Madrid]{
		\includegraphics[width=0.74\textwidth]{4-Madrid-AirQuality.png}
		\label{fig:4-Madrid-AirQuality}
	}
	\subfigure[Berlin]{
		\includegraphics[width=0.74\textwidth]{4-Berlin-AirQuality.png}
		\label{fig:4-Berlin-AirQuality}
	}
	\subfigure[Paris]{
		\includegraphics[width=0.74\textwidth]{4-Paris-AirQuality.png}
		\label{fig:4-Paris-AirQuality}
	}
	\caption{Air Quality Details on 6th February 2018 at 18:00}
	\label{fig:4-AirQuality-Details}{Source: Air Quality Now webpage \cite{airqualitynow}}
\end{figure}



\section{Embedded systems}
% Sistemas Empotrados

\subsection{Embedded systems in traffic or environmental surveillance}
% Buscar sistemas embebidos de control de tráfico o medio ambiental.
%	- https://acuraembedded.com/blog/acura-embedded-systems-to-reduce-carbon-emissions-for-public-transport-buses/
%	- ...
% MENCIONAR:
% Estas estaciones solo mide un punto de la ciudad, nosotros buscamos tecnologías baratas y que permitan controlar varios puntos
% Masivo

\section{Raspberry Pi embedded system}
% Raspberry Pi 3 	-> Evolución
%					-> Comparativa con sus competidores
% 					-> Por qúe se ha elegido?
% !! Muy interesante:  https://www.raspberrypi.org/blog/vectors-from-coarse-motion-estimation/
%					->  Raspbian




\section{Video format H.264/AVC}
\label{subsect:H.264}
Any video is composed by a sequence of images that are reproduced in a sequential way. In H.264/AVC video format and some others video standards each of this video images (or frames) can be classified into three types: I-Frame, P-Frame and B-Frame \cite{SC11}. Each of this frame types take advantage of different type of spatial or temporal redundancy of the video sequence: 
\begin{itemize}
	\item \textbf{I-Frame (Inter-frame).} The frame is encoded using only spatial redundancy inside the frame itself.
	\item \textbf{P-Frame (Predictive-frame).} The frame is encoded using as reference a previous I- or P- frame (temporal redundancy).
	\item \textbf{B-Frame (Bi-Predictive-frame).} The frame is encoded using more than one reference image (previous and past frames).
	\item \textbf{Other types}. The extended format of the H.264/AVC decoder support more complex types: SP (Switching P-frame) and SI (Switching I-frame).
\end{itemize}

\subsection{Macroblocks}
\label{subsect:Macroblocks}
A \emword{macroblock} is the basic unit in the video compression formats based on linear block transforms. It contain the information of a 16x16 pixels region of the frame. These blocks contains information about the luminance, the chrominance and the \emword{motion vectors} associated to them. Each macroblock can be divided into several block of lower size called partitions \cite{Gir14}, which varies from 16x16 to 4x4, as shown in Figure \ref{fig:4-Macroblocks}. In this figure, it is shown how a macroblock can be divided into two partitions of 16x8 or 8x16, our just in four partitions of 8x8, and each of this 8x8 partitions can be divided into two partitions of 8x4 or 4x8 or into four partitions of 4x4.

\begin{figure}[!h]
	\begin{center}
		\includegraphics[width=0.8\textwidth]{4-Macroblocks.png}
		\caption{Macroblock partitions.}
		\label{fig:4-Macroblocks}
	\end{center}
\end{figure}


\subsection{Motion vectors}
A motion vector represent a temporal redundancy pattern between two frames in H.264/AVC and defines a distance and a direction in the form of a bidimensional vector. In other words, they represent the movement of a certain macroblock in the actual image with respect to the reference image. Figure \ref{fig:4-Car_with_MV} shows an example of the motion vectors generated by a Raspberry Pi while recording video in a street. Therefore, instead of storing all the pixels for the current frame, the motion vectors corresponding with the macroblocks that have been identified in the reference image are stored, which saves lot of space when storing the video.

PiCamera motion data values for each macroblock are 4-bytes long, as it consist in 1-byte $x$ vector, 1-byte $y$ vector and two 2-byte \ac{SAD} value. If, for example, the resolution of the video is 640x480, there will be 41 columns of macroblocks (640 pixels / 16 pixels per macroblock + 1, as PiCamera generates one extra column), and 30 rows (480 pixels / 16 pixels per macroblock). Therefore, each frame generates $41\times30\times4 = 4920$ bytes, which is less than 5KB of motion data.

\begin{figure}[!h]
	\begin{center}
		\includegraphics[width=0.8\textwidth]{4-Car_with_MV.png}
		\caption{Example of  the \emword{motion vectors} taken by the Raspberry Pi}
		\label{fig:4-Car_with_MV}
	\end{center}
\end{figure}

Motion vectors can be represented using its Cartesian coordinates as shown in Equation (\ref{eq:4-mv-cartesian}), where $x$ and $y$ represents the original position of the macroblock in the first frame and $I_{x}$ and $I_{y}$ represents the increments of the vector in each of its coordinates.
\begin{equation} \label{eq:4-mv-cartesian}
mv_{i} = (x, y, I_{x}, I_{y})
\end{equation}
Nevertheless, the motion vectors can also be represented usign polar coordinates $r$ and $\theta$ \cite{GRMSJ12}. Therefore, each motion vector can be represented as shown in Equation (\ref{eq:4-mv-polar}).
\begin{equation} \label{eq:4-mv-polar}
mv_{i} = (x, y, r, \theta)
\end{equation}




% ESTO NO:
% Un párrafo hablando de los vectores de movimento y Codificación de vídeo -> H.264/AVC, lincado a los sistemas embebidos 
%	- https://www.raspberrypi.org/blog/vectors-from-coarse-motion-estimation/


