% TFG - José Ángel Martín Baos. Escuela Superior de Informática. 2018
%%%% CHAPTER: Conclusions %%%
\chapter{Conclusions}
\label{chap:conclusions} 

\drop{T}{his} chapter presents a summary of the different goals achieved during the execution of this \ac{BSc.} thesis. Moreover, it proposes possible future works and ideas to extend the project.


\section{Goal achievements}

The main objective of this \ac{BSc.} thesis (defined in Section \ref{chap:main-objective}) was the design and development of a pollution and traffic surveillance system prototype. With its construction it has been demonstrated the technical viability of the system.

The main goal has been achieved through the completion of the four specific objectives planned at the initial stage of the project. Now the satisfaction of the different specific objectives is analysed.
\begin{itemize}
	\item \textbf{O.1.} This objective has been fulfilled during the Sprint 1. For the completion of this objective, an efficient algorithm to determine the traffic flow in a street was developed. This algorithm detect correctly about the 90\% of the vehicles, which makes this solution compatible with the objective pursued in the \ac{BSc.} thesis. Thanks to the use of motion vectors, which are generated by the H.264/AVC video encoder on the \ac{GPU}, the number of vehicles that circulates on a street was counted in real time consuming low \ac{CPU} resources (about the 10\% of the \ac{CPU}). The algorithm is executed approximately 14 times faster than the time available between frames. Therefore, the use of this approach allows to use an inexpensive embedded system such as the Raspberry Pi.
	
	\item \textbf{O.2.} The fulfilment of this objective was achieved during the Sprint 2. An electronic circuit was developed to integrate different sensors. Firstly, the correct integration of several environmental sensors such as temperature, humidity and pressure sensors was achieved. These sensors are very important in order to calculate the pollutants dissipation in some emission models. Secondly, the integration of the LPG sensor was completed. Nonetheless, the integration of the CO gas sensor presents a minor problem: the electrical current needed to heat the sensor and dissipate the CO between samples is not the recommended one (5 volts). The reason is that the voltage regulator chip (LM317) has an internal consumption and, therefore, the 5 volts obtained from the Raspberry Pi device were decreased by this chip. A feasible solution is to connect this chip to an external power supply with a higher voltage, for instance 9 volts.
	
	\item \textbf{O.3.} This objective was completely fulfilled during the execution of the Sprint 3. Through the use of the IBM Watson IoT Platform the scalability property is added to the solution developed in this thesis, which allows to use the device massively on urban areas. This is a key goal of this project and \ac{IoT} systems in general.
	
	\item \textbf{O.4.} This objective was completed during the Sprint 4. A web page was built in order to allow the visualization of the data generated by every connected device. Moreover, this web page has been developed taking into account the affordance, usability, and visibility of the design in order to ease its use.
\end{itemize}


\section{Future work}
In this section some ideas to improve the prototype and to expand it are proposed. In addition, other domains where the development project can be applied are described.

\subsection{Prototype improvements}
Some improvements can be done to add additional value to the prototype developed in this \ac{BSc.} thesis. The improvements may be the following:
\begin{itemize}
	\item An automatic parameter tuning of the vehicle detection algorithm can be developed. Using different machine learning and video analysis techniques the different thresholds and variables of the vehicle counting algorithm developed in Sprint 1 can be estimated. Therefore, these values can be calculated automatically when the device is placed in a new location, saving the time needed for trial and error estimation. 
	
	\item The correct calibration of the different gas sensors. The best method consist in locating the sensors into a closed box with a known concentration of the target gas, therefore, the sensor can be calibrated with that concentration. The recommended concentration required for the correct calibration can be found in the sensors documentation. 
	
	\item In this project it has been demonstrated that we can integrate any gas sensor into the developed device. Therefore, more pollutant gases (such as NO\textsubscript{2}, PM\textsubscript{10}, PM\textsubscript{2.5} or  SO\textsubscript{2}) can be measured by installing other type of sensors.
	
\end{itemize}


\subsection{Prototype expansion}
The developed device can be used as a basis for the design of a \ac{ITS}, such as the one proposed in Figure \ref{fig:7-Prototype_expansion_diagram}. In order to design an intelligent system to predict the pollutants concentrations, a mathematical emission model, such as the COPERT \cite{NS16} model, could be integrated into the system. This model would estimate the emissions according to the type of vehicle, the speed, and the operating time. Six different types of vehicles are defined by the model and, therefore, the type of the different vehicles counted should be determined. Two strategies could be used to determine the vehicle type: The first one consists in capturing the license plates of the vehicles that are detected and use this information to obtain some relevant data such as the type of vehicle. The second strategy consists in using an image analysis technique, such as edge detection along with a deep learning algorithm to determine the type of vehicle.

This system can recommend traffic actions to reduce the contamination predicted by the COPERT model. Some actions that can be taken by the system could be: closing the traffic or lowering the speed limits in some streets, restricting the traffic in some areas to a certain number of vehicles or to certain license plates, etc. In addition, the system can measure the improvements generated by the actions taken, obtaining some feedback. This feedback can be used by a reinforcement learning algorithm to improve future traffic control decisions of the system.

\begin{figure}[!h]
	\begin{center}
		\includegraphics[width=0.94\textwidth]{7-Prototype_expansion_diagram.pdf}
		\caption{Prototype of the proposed Intelligent Transport System}
		\label{fig:7-Prototype_expansion_diagram}
	\end{center}
\end{figure}


\subsection{Other domains where the project can be used}
The device developed in this \ac{BSc.} thesis can also be applied to other domains. For example, the device can be adapted to be used as a control system in factories. Then, the gas sensors can be used to determine the concentration of some gases that could be dangerous to humans. These gases can be produced due to the normal operation of some machines or can be caused by a gas leak. Therefore, it is important to monitor the air of some factories to avoid an overexposure to toxic gases. Moreover, the camera can be used to count the number of units produced by the machines and to determine, for example, if any unit has a defect.

Another example is to adapt the developed camera system to count pedestrians in public areas or travellers in public transports such as trains. These can be used to monitor the number of people in public areas or determine the people flow inside rail stations.


\quoteauthor{Ciudad Real, 27th June 2018}
\quoteauthor{José Ángel Martín Baos}



%%%% CHAPTER: Conclusiones %%%
\chapter{Conclusiones}
\label{chap:conclusiones}

\drop{E}{n} este capítulo se presenta un resumen de los diferentes objetivos logrados durante la realización del \ac{TFG}. Además, se proponen posibles trabajos futuros e ideas para ampliar el proyecto.

\section{Análisis de la consecución de los objetivos}
El objetivo principal de este \ac{TFG} (definido en la Sección \ref{chap:main-objective}) es el diseño y desarrollo de un prototipo de sistema de vigilancia de la contaminación y del tráfico en áreas urbanas. Con este trabajo se ha demostrado que el sistema propuesto es técnicamente viable.

El objetivo principal se ha logrado mediante la realización de cuatro objetivos específicos planificados en la primera etapa del proyecto. A continuación, el grado de consecución de los diferentes objetivos específicos será analizada.

\begin{itemize}
	\item \textbf{O.1.} Este objetivo ha sido cumplido durante la realización del Sprint 1. Para ello, se ha desarrollado un algoritmo eficiente capaz de determinar el flujo de tráfico. Este algoritmo detecta aproximadamente el 90\% de los vehículos correctamente, lo que lo convierte en una solución compatible con el objetivo perseguido en el \ac{TFG}. Mediante el uso de vectores de movimiento, generados en la \ac{GPU} usando el codificador de video H.264/AVC, el número de vehículos que circulan por una calle pueden ser contados en tiempo real y consumiendo bajos recursos de la \ac{CPU} (aproximadamente el 10\%). Este algoritmo se ejecuta aproximadamente 14 veces más rápido que el tiempo del que se dispone entre frames. Por lo tanto, el uso de este enfoque permite el uso de un sistema empotrado de bajo coste como la Raspberry Pi.
	
	\item \textbf{O.2.} Este objetivo fue logrado durante el Sprint 2. Se ha desarrollado un circuito electrónico para integrar los diferentes sensores. En primer lugar, se ha logrado la correcta integración de diferentes sensores ambientales como de temperatura, humedad y presión. Estos sensores son muy importantes para poder calcular la disipación de los contaminantes mediante modelos de emisiones. En segundo lugar, la integración del sensores de LPG y CO se ha realizado correctamente. Sin embargo, la integración del sensor de CO presenta un problema: la corriente necesaria para calentar el sensor y disipar el CO entre mediciones no es la recomendada (5 voltios). Esto se debe a que el chip regulador de tensión (LM317) tiene una deriva interna y, por lo tanto, los 5 voltios obtenidos del dispositivo Raspberry Pi no son alcanzados en el sensor. Una posible solución es conectar este chip a una fuente de electricidad con un voltaje superior, por ejemplo 9 voltios. 
	
	\item \textbf{O.3.} Este objetivo ha sido completado durante la ejecución del Sprint 3. Mediante el uso de IBM Watson IoT Platform se ha añadido la propiedad de escalabilidad a la solución desarrollada en este trabajo, lo que permite el uso del dispositivo desarrollado de manera masiva en zonas urbanas. Este es un objetivo clave de este proyecto y de los sistemas \ac{IoT} en general.
	
	\item \textbf{O.4.} Este objetivo fue completado durante el Sprint 4 mediante la realización de una página web que permite la visualización de los datos generados por cada dispositivo conectado. Además, esta página web ha sido desarrollada teniendo en cuenta el \textit{affordance}, la usabilidad y la visibilidad del diseño para facilitar su uso.
\end{itemize}


\section{Trabajo futuro}
En esta sección se proponen algunas ideas para mejorar y expandir el prototipo propuesto. Además, se describen otros dominios donde el proyecto desarrollado puede ser aplicado.

\subsection{Mejora del prototipo}
Algunas mejoras pueden ser realizadas para aportar un valor adicional al prototipo desarrollado en este \ac{TFG}. Estas mejoras pueden ser las siguientes:

\begin{itemize}
	\item Desarrollo de un sistema de calibración automática de los parámetros del algoritmo de detección de vídeo. Usando diferentes técnicas de machine learning y análisis de vídeo se pueden calcular los diferentes umbrales y variables necesarias para el algoritmo de conteo de vehículos desarrollado en el Sprint 1. Por lo tanto, estos valores pueden ser calculados automáticamente cuando el dispositivo se coloca en una nueva localización, ahorrando el tiempo necesario para una calibración mediante prueba y error.
	
	\item Calibrar los sensores de gas de manera precisa. El mejor método para lograrlo consiste en colocar el sensor en un recipiente cerrado con una concentración de gas previamente conocida, de forma que el sensor se pueda calibrar usando esa concentración. En la documentación de los sensores se encuentra detallada la concentración recomendada para la correcta calibración de cada sensor.
	
	\item En este \ac{TFG} se ha demostrado que se puede integrar cualquier sensor de gas en el dispositivo desarrollado. Además, se pueden medir más tipos de gases contaminantes mediante la instalación del sensor correspondiente. Ejemplos de gases que podrían medirse son el NO\textsubscript{2}, PM\textsubscript{10}, PM\textsubscript{2.5} o  SO\textsubscript{2}.
	
\end{itemize}


\subsection{Expansión del prototipo}
El dispositivo desarrollado puede ser usado como base para el diseño de un Sistema Ingeligente de Transporte, como el propuesto en la Figura \ref{fig:8-Diagrama_expansión_prototipo}. Para poder desarrollar un sistema inteligente que sea capaz de predecir la concentración de los contaminantes se necesita integrar un modelo matemático de emisiones, como el modelo COPERT \cite{NS16}. Este modelo estima las emisiones dependiendo del tipo de vehículo, su velocidad y el tiempo de funcionamiento. En el modelo se definen seis tipos diferentes de vehículos, por lo tanto se debe determinar el tipo de cada vehículo detectado. Para esto se pueden usar dos estrategias: La primera consiste en capturar las matrículas de los vehículos que son detectados y usar esta información para obtener información relevante, como por ejemplo el tipo de vehículo. La segunda estrategia consiste en usar técnicas de análisis de imagen, como por ejemplo detección de bordes, junto con un algoritmo de \textit{deep learning} para determinar el tipo de vehículo.

Este sistema puede recomendar acciones de tráfico para reducir la contaminación predicha por el modelo COPERT. Algunas de las acciones que podría  recomendar el sistema son: cerrar el tráfico o disminuir los límites de velocidad en algunas calles, restringir el tráfico en algunas áreas a un número determinado de vehículos o por un determinado subconjunto de matrículas, etc. Además, el sistema podría medir el efecto generado por cada una de las acciones tomadas, obteniendo de esta manera una retroalimentación. Esta información puede ser usada por un algoritmo de aprendizaje por refuerzo para mejorar la futura toma de decisiones de control de tráfico.

\begin{figure}[!h]
	\begin{center}
		\includegraphics[width=0.94\textwidth]{8-Diagrama_expansion_prototipo.pdf}
		\caption{Prototipo del Sistema Inteligente de Transporte propuesto}
		\label{fig:8-Diagrama_expansión_prototipo}
	\end{center}
\end{figure}


\subsection{Otros dominios donde puede ser aplicado el proyecto}
El dispositivo desarrollado en este \ac{TFG} puede ser aplicado también a otros dominios. Por ejemplo, el dispositivo puede ser adaptado para ser usado como sistema de control en algunas fábricas. De esta manera, los sensores de gas se pueden usar para determinar la concentración de algunos gases que pueden ser peligrosos para los humanos. Estos gases pueden ser producidos mediante el funcionamiento normal de algunas máquinas o debidos a fugas de gas. Por lo tanto, es muy importante monitorizar el aire de algunas industrias de forma que se evite una sobreexposición de los trabajadores a gases tóxicos. Además, la cámara puede ser usada para contar el número de unidades producidas por las máquinas, o incluso para determinar si alguna unidad tiene un defecto.

Otro ejemplo es adaptar el módulo de la cámara desarrollado para contar peatones en zonas públicas o viajeros en medios de transporte como trenes. De esta manera se puede controlar el número de personas en determinadas zonas públicas o incluso determinar el flujo de viajeros dentro de estaciones de tren.

\quoteauthor{Ciudad Real, a 27 de Junio de 2018} 
\quoteauthor{José Ángel Martín Baos}

