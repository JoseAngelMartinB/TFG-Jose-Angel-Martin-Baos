% TFG - José Ángel Martín Baos. Escuela Superior de Informática. 2018
%%%% CHAPTER: Conclusions %%%
\chapter{Conclusions} %TODO
\label{chap:conclusions} 

\drop{T}{his} chapter presents a summary of the different goals achieved during the execution of this \ac{BSc.} thesis. Moreover, this chapter proposes some future works and ideas to extend the project done in this thesis.


\section{Goal achievements}

The main objective of this \ac{BSc.} thesis (defined in Section \ref{chap:main-objective}) has been the design and development of a pollution and traffic surveillance system prototype. With its construction it has been demonstrated the technical viability of the system.

This main objective has been achieved through the completion of the four specific objectives planed at the beginning of the project. Now the completion different specific objectives are analysed.
\begin{itemize}
	\item \textbf{O.1.} This objective has been fulfilled during the Sprint 1. For the completion of this objective, an efficient algorithm to determine the traffic flow in a street has been developed. This algorithm has obtained a hit rate of 90\%, which makes this solution compatible with the objective pursued in the project. Thanks to the use of motion vectors, which are generated by the H.264/AVC video encoder on the \ac{GPU}, the number of vehicles that travels on a street have been counted in real time and consuming low CPU resources. The algorithm is executed approximately 14 times faster than the time available between frames. Therefore, the use of this approach, which requires very low computational resources, allows to use an inexpensive embedded system such as the Raspberry Pi.
	
	\item \textbf{O.2.} For the completion of this objective, achieved during Sprint 2, an electronic circuit has been developed to integrate different sensors. Firstly, the correct integration of several environmental sensors such as temperature, humidity and pressure sensors has been achieved. These sensors are very important in order to calculate the pollutants dissipation in some emission models. 
	
	Secondly, the integration of the LPG sensor has been completed without problems. Nonetheless, the CO gas sensor has been integrated but it presents a minor problem: the current needed to heat the sensor and dissipate the CO between measures is not the recommended one (5 volts). The reason is that the voltage regulator chip (LM317) has an internal consumption and, therefore, the 5 volts obtained from the Raspberry Pi device are decreased by this chip. A feasible solution is to connect this chip to an external power supply with a higher voltage (for instance 9 volts).
	
	\item \textbf{O.3.} This objective has been completely fulfilled during the execution of Sprint 3. Through the use of the IBM Watson IoT Platform the solution developed in this thesis is scalable, which allows to use the device developed massively on urban areas. This is a key goal of this project and has been achieved.
	
	\item \textbf{O.4.} This objective was completed during Sprint 4. A web page has been developed to visualize the data generated by the different devices. Moreover, this web page has been developed taking into account the affordance, usability, and visibility of the design in order to ease its use.
\end{itemize}


% Trabajo futuro
\section{Future work}
\subsection{Prototype improvements}
Some improvements can be done to add value to the prototype developed in this \ac{BSc.} thesis. The improvements may be the following:
\begin{itemize}
	\item An automatic parameter tuning of the vehicle detection algorithm can be developed. Using different machine learning and video analysis techniques the different thresholds and variables of the vehicle counting algorithm developed in Sprint 1 can be estimated. Therefore, these values can be calculated automatically when the device is placed in a new location, saving the time needed for a trial and error estimation. 
	
	\item The correct calibration of the different gas sensors. The best method consist in locating the sensors into a closed box with a known concentration of the target gas, therefore, the sensor can be calibrated with that concentration. The recommended concentration used for the calibration can be found in the sensors documentation. 
	
	\item In this project it has been demonstrated that we can integrate any gas sensor into the developed device. Therefore, more pollutant gases (such as NO\textsubscript{2}, PM\textsubscript{10}, PM\textsubscript{2.5} or  SO\textsubscript{2}) can be measured by installing other type of sensors.
	
\end{itemize}


\subsection{Prototype expansion}
The developed device can be used as a basis for the design of a intelligent transport system, such as the one proposed in Figure \ref{fig:7-Prototype_expansion_diagram}. In order to design an intelligent system able to predict the pollutants concentrations, an emission mathematical model, such as the COPERT \cite{NS16} model, has to be integrated into the system. This model estimates the emissions according to the type of vehicle, the speed, and the operating time. Six different types of vehicles are defined by the model and, therefore, the type of the different vehicles counted has to be determined. Two strategies can be used to determine the vehicle type: The first one consists in capturing the number plates of the vehicles that passes and use this information to obtain some relevant data such as the type of vehicle. The second strategy consists in using an image analysis technique, such as edge detection, along with a deep learning algorithm to determine the type of vehicle.

This system can recommend traffic actions to reduce the contamination predicted by the COPERT model. Some actions that can be taken by the system are: closing the traffic or lowering the speed limits in some streets, restricting the traffic in some areas to certain number of vehicles or to certain number plate, etc. In addition, the system can measure the improvements generated by the actions taken, obtaining some feedback. This feedback can be used by a reinforcement learning algorithm to improve future traffic control decisions of the system.

\begin{figure}[!h]
	\begin{center}
		\includegraphics[width=1\textwidth]{7-Prototype_expansion_diagram.pdf}
		\caption{Prototype of the proposed intelligent transport system}
		\label{fig:7-Prototype_expansion_diagram}
	\end{center}
\end{figure}


\subsection{Other domains where the project can be used}
The device developed in this \ac{BSc.} thesis can also be applied to other domains. For example, the device can be adapted to be used as a control system in factories. Then, the gas sensors can be used to determine the concentration of some gases that could be dangerous to humans. These gases can be produced with the normal operation of some machines or can be caused by a gas leak. Therefore, it is important to monitor the air of some factories to avoid an overexposure to toxic gases. Moreover, the camera can be used to count the number of units produced by the machines and to determine if any unit has a defect.

An other example is to adapt the developed camera system to count pedestrians in public areas or travellers in public transports such as trains. These can be used to monitor the number of people in public areas or determine the people flow inside rail stations.


\quoteauthor{Ciudad Real, 27th June 2018} %TODO
\quoteauthor{José Ángel Martín Baos}



%%%% CHAPTER: Conclusiones %%%
\chapter{\REDNOTE{Conclusiones}} %TODO
\label{chap:conclusiones}

% \drop{I}{n}


% Ideas:
% Reflexionar sobre los principales hitos conseguidos. Comparativa objetivos planteados vs objetivos consegudios
% Plantear trabajo futuro -> Posibles mejoras o amplificaciones del TFG
%	- Mejora de los Sensores. Mejora de los circitos para el voltaje, mejora de la calibración, ...
% Reflexión sobre los conocimientos adquiridos ?


\quoteauthor{Ciudad Real, a 27 de Junio de 2018} %TODO
\quoteauthor{José Ángel Martín Baos}

